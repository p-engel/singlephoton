\documentclass[12pt]{article}
\pagestyle{plain}


\begin{document}

\title{PhD Candidacy Exam Proposal:}
\author{c praise anyanwu \\
    Chemistry Department, University of Washington
    \thanks{anyanc@uw.edu} }
\date{14 September 2022}
\maketitle

\newpage

\section{A Single Photon Source Enabled by Fano Interference between a
Broadband Emitter and a High Q, Kerr Cavity}

Single photon sources play a prominent role in the field of quantum information 
science. \cite{gisin2002quantum, bennett2004quantum} In the QKD protocol, Alice encodes 
an information, for example, in the quibit of a photon polarization state. If Alice must 
securely transmit this quibit through a quantum channel to Bob, the reciever, then 
Alice's single photon source must be deterministic. In a case where Alice generates a 
probalistic two-photon state, then Eve, the eavesdropper, can glean information from one 
of the two photons (unbeknownst to Alice) while the second photon is transmitted to Bob. 
\cite{bennett1984proceedings, bennett1992quantum} Moreover, since photons travel at the 
speed of light and interact weakly with the environment over long distances, enconding 
infromation in the quanutm state of a single photon (using degrees of freedom of 
polarization, momentum, or energy) is hihgly compatible and thus desirabble in 
quantum communication application.

A deterministic single-photon source that emits a single photon \textit{on demand}, 
with $100 \: \%$ probability, is ideal. In practice, one evaluates the 
single-photon nature of a source by the ratio of the probability of single-photon 
to multi-photon emission, and thus single-photon sources lie on a spectrum of 
two main classes: deterministic and probabilistic sources. \cite{lounis2005single, 
eisaman2011invited} The former involves effective two level systems (quantum
dots, single atoms, single ions) \cite{ShieldsAndrewJ2007Sqls, strauf2007high, 
hennrich2004photon, wilk2007single, maurer2004single} that emit a single photon when 
excited by a resonant incident field; the latter involves, for example, parametric 
down conversion in waveguides (or four-wave mixing in optical fiber) systems
\cite{u2004efficient, sharping2001observation, goldschmidt2008spectrally} 
that emit a correlated pair of photons, where one photon heralds the other. 
The known difficulties with these systems involve trapping of a single ion or atom 
strongly coupled to cavities at cryogenic temperatures for deterministic single-photon 
sources, and care must be taken with the probabilistic sources to avoid generating 
multiple pairs of photons. Herein, we propose a theoretical basis for a novel system that 
circumvents ion trapping at ultra-cold temperatures, and this system is operative in the 
Purcell regime; then we calculate the second order correlation function as a measure of 
its single-photon nature.

Consider the following system:

\newpage
\bibliography{ref_database}
\bibliographystyle{unsrt}

\end{document}
