\documentclass[12pt]{article}
\pagestyle{plain}
\usepackage{amsmath}
\usepackage{bm}
\usepackage{graphicx}
\graphicspath{ {./} }
\usepackage{float}

\begin{document}

\title{PhD Candidacy Proposal: \\
A Strategy for the Creation of a Single Photon Using a Kerr-Cavity Weakly
Coupled to Quantum Emitters}
\author{c praise anyanwu \\
    Chemistry Department, University of Washington
    \thanks{anyanc@uw.edu} }
\date{14 September 2022}
\maketitle

\newpage
\section{Introduction}
Since photons travel at the speed of light and interact weakly with the
environment over long distances, enconding infromation in the quanutm state of
a single photon (using degrees of freedom of polarization, momentum, or energy)
is highly desirable in secure quantum communication application.
\cite{gisin2002quantum, bennett2004quantum} For example, in the quatum key 
distribution protocol, Alice encodes an information in the quibit of a photon 
polarization state. \cite{bennett1984proceedings, bennett1992quantum}
Importantly, if Alice must securely transmit this quibit through a quantum channel 
to Bob, and without loss of information to an eavesdropper, then Alice's 
single-photon source must be deterministic. A deterministic single-photon 
source emits a single photon \textit{on demand}, with $100 \: \%$ probability. 
In practice, one evaluates the single-photon nature of a source by the ratio of 
the probability of single-photon to multi-photon emission. \cite{lounis2005single}

Single-photon sources lie on a spectrum of two main classes.
\cite{lounis2005single, eisaman2011invited} The former lies towards a
deterministic source. It involves effective two level systems (quantum dots, 
single atoms, single ions) \cite{ShieldsAndrewJ2007Sqls, strauf2007high, 
hennrich2004photon, wilk2007single, maurer2004single} that emit a single photon 
when excited by a resonant incident field. The latter sources are less 
deterministic and involve such processes as parametric down conversion in 
waveguides, or four-wave mixing in optical fiber systems 
\cite{u2004efficient, sharping2001observation, goldschmidt2008spectrally} 
that emit a correlated pair of photons, where one photon heralds the other. 
The known difficulties with these systems include trapping of a single ion or 
atom strongly coupled to cavities at cryogenic temperatures for deterministic 
single-photon sources, while care must be taken with the less deterministic 
sources in order to avoid generating multiple pairs of photons. Herein, 
we propose a theoretical basis for a novel single-photon source that circumvents 
ion-trapping at ultra-cold temperatures, and this system is operative in the 
Purcell regime; then we calculate the second order correlation function as 
a measure of its single-photon nature.

Consider the following system: a high-Q cavity coupled to a broad-band emitter.
For example, ref \cite{pan2019elucidating} has an emitter placed on the chip of 
a torodial silica based microcavity with a $10^{7}$ Q-factor (Fig. 1a). 
We propose the following modification: a quantum emitter placed on a 
kerr-cavity coated with polystyrene, which has a significant 
$10^{- 12} \:\mathrm{cm}^2/\mathrm{W}$ third-order nonlinear susceptibility.
\cite{qin2010design, liu200910} (Polystyrene's third-order nonlinearity 
originates from the delocalization of the $\pi$-conjugated electrons along the 
polymer chains. \cite{krausz1989optical, wong1991studies}) This achieves a kerr 
effect such that the resonance of the nonlinear cavity now depends on the 
photon-number in the cavity (Fig. 1b).

\begin{figure}[]
  \centering
  \includegraphics[width=1\linewidth]{fig.pdf}
  \caption{(a) Example of the proposed experimental set-up
  \cite{pan2019elucidating}, which in general, consists of a broad-band
  quantum emitter (here represented as gold nanorod) placed unto\textemdash
  and weakly coupled to\textemdash a kerr-cavity (here represented as
  toroidal micro-ring resonator). The emitter is pumped with a
  Laser, and its absorption spectrum shows a Fano profile due
  to its interaction with the High-Q cavity. At the vicinity of the
  anit-resonance, there is a peak in tramission of light through the cavity
  out-coupled to the measuring fiber optic cable. (b) The spectral distribution
  of the reduced average photon-number in the broad-band mode at steady-state
  $<\hat{b}^{\dagger}\hat{b}>_{ss}/<\hat{b}^{\dagger}\hat{b}>_{ss; g=0}$,
  yellow curves, and the average photon number in the discrete cavity mode of a
  single-photon state $<n_c=1|\: \hat{a}^{\dagger}\hat{a} \: |n_c=1>_{ss}$, 
  blue curves. The solid curves are the spectra when the cavity has zero-photon
  occupation $N_c=0$, while the dashed curves, the cavity has a single-photon
  occupation $N_c=1$. (c) The second order correlation calculated at steady
  state when the cavity is populated with a single-photon occupation.
  The single-photon blockade is eviedent for a kerr-nonlinearity whose value is
  withtin the order of the Fano line-width, i.e., $U=\Gamma$.
  }
  \label{fig}
\end{figure}

A characteristic of the coupled kerr-cavity-emitter system is the cavity 
photon-number dependent Fano resonance, with a sharp resonant peak and 
anti-resonant dip corresponding to an enhanced and diminished absorption by the
emitter (see Fig. 1b). Just at the vicinity of the  anti-resonance, there is a 
transmision of photon into the cavity. Upon single-photon population, the
kerr-cavity resonance is detuned from a pump laser exciting the system at the
anti-resonance. And if the kerr-cavity resonant frequency shift is on the order 
$\Gamma$ (the FWHM equal to the line-width between the anti-resonant dip and 
the resonant peak of the Fano Profile), then the system experiences a 
single-photon blockade, as shown in the calculated $g^{(2)}$ (Fig. 1c) of light 
transmitted through the cavity. Thus the single-photon blockade effect depends 
on the Fano resonance shift $\Delta\omega$ induced by the Kerr-cavity's 
non-linearity, and the degree of nonlinearity is proportional to $\Gamma$, 
such that,
\begin{equation}
\begin{split}
\omega_1 - \omega_0 &= (\omega_0 + \Gamma ) - \omega_0
\\
\Delta\omega &= \Gamma
\end{split}
\end{equation}
where $\omega_{0,1} = c n_{0,1} k$, and $n_1$ is the refractive index due
to the kerr effect \cite{spillane2002ultralow}:
\begin{equation}
\begin{split}
n_1 &= n_0 + n_2 I
\\
&= n_0 + \frac{3 \chi^{(3)}}{8 n_0 c\epsilon_0} I.
\end{split}
\end{equation}
Note that $n_2$ is the value often reported for the third-order suceptibility,
and the kerr effect is induced by the local intensity $I$ from the pump laser 
populating the kerr-cavity. From the above equations we find the relation:
\begin{equation}
\begin{split}
\frac{\Delta \omega}{ck} &= n_2 I \\
\end{split}
\end{equation}
For a cavity mode $\lambda/{n_0} \approx 1550 \:\mathrm{nm}$, with a 
third-order susceptibility $n_2 \approx 1.15 \times 10^{-12}
\:\mathrm{cm}^2/\mathrm{W}$, \cite{qin2010design} the desired blockade effect 
can be achieved for a single-photon state whose resonance is shifted by
$\Delta\omega \approx \Gamma$, if the local intensity $I \ge $, which is
the regime of a faint laser beam whose coherent state could have an average 
photon-number in the range $\mu = 1-4$. This is advantageous because the 
desired blockade effect could be achieved with a faint laser beam or with a 
kerr-cavity of much smaller third-order suceptibily.


\section{Model}
We develop the system's dynamics in Heisenberg picture. Heisenberg picture is 
suitable to calculate the quantum statistical correlation of the field of light 
transmitted throught the kerr cavity. This field correlation can be related to 
the experimentally observed spectral distribution, using the spectral response 
function. \cite{scully1999quantum} Moreover, in the Heisenberg picture the 
dynamics of the quantum transverse field is analogous to the classical 
transverse field \cite{tannoudji1992atom}, which allows for direct comparison 
to the classical system. \cite{pan2019elucidating} Then damping of the 
single-mode in the cavity field (and of the broad-band field) is described 
using Heisenberg-Langevin formalism. In so doing, this model provides a simple 
yet rigorous quantum approach (complementing the Louivillian formalism in the
Linblad form, recently developed in Ref. \cite{finkelstein2015fano, 
finkelstein2016nonlinear}) to account for damping of the discrete state 
(cavity mode) of Fano resonance, and the spectral distribution derived from 
this model is expressed in terms of Fano-parameters. \cite{fano1961effects}

The Hamiltonian of the discrete kerr-cavity coupled to a broad-band emitter is
the following $\mathcal{H} = \mathcal{H}_{a} + \mathcal{H}_{b} + 
\mathcal{H}_{f} + \mathcal{H}_{I}$. The derived Hamiltonian $\mathcal{H}_{a}$ 
is the energy of a single-mode dielectric kerr-cavity in free space 
\cite{jackson1999classical}, such that,
\begin{equation}
\begin{split}
\mathcal{H}_{a} &= \frac{1}{2} 
    \int_{V} \mathrm{d}^{3}\mathbf{x} \: \mathbf{P} \cdot \mathbf{E}
\\
&= \frac{1}{2} 
    \int_{V} \: \mathrm{d}^{3}\mathbf{x}
    \Big( \mathbf{P}^{(1)} + \mathbf{P}^{(3)} \Big)
    \cdot \mathbf{E}
\\
&= \frac{1}{2} 
    \int_{V} \: \mathrm{d}^{3}\mathbf{x} \: 4\pi 
    \Big( \bm\chi^{(1)} \mathbf{E}  + 
    \frac{3}{4}\bm\chi^{(3)}:\mathbf{E}\mathbf{E}^{*} \mathbf{E} \Big)
    \cdot \mathbf{E}
\end{split}
\end{equation}
where $3/4\mathbf{\chi}^{(3)}$ is the higher order non-linear suceptibility of
a kerr-cavity \cite{butcher1990elements}, and the quantized field $\mathbf{E}$
is normalized with respect to the cavity's mode volume $V$. In the rotating
wave approximation (and for a linear polarization), $\mathcal{H}_{a}$
simplifies as follows
\begin{equation}
\begin{split}
\mathcal{H}_{a} &= \hbar\omega_k 4\pi^{2} \Big(
    \chi^{(1)}_{ef} +
    \frac{3}{4} \chi^{(3)}_{ef} \mathcal{E}^2
    a^{\dagger}a + \frac{3}{4}\bm\chi^{(3)} \mathcal{E}^2 : \mathbf{1} \Big)
    \Big( a^{\dagger}a + \frac{1}{2} \Big)
\\
&= \hbar \Big( \omega_c + U\:a^{\dagger}a \Big)
    \Big( a^{\dagger}a + \frac{1}{2} \Big)
\\  
&\approx \hbar \Big( \omega_c + U \langle a^{\dagger}a \rangle \Big)
    \Big( a^{\dagger}a + \frac{1}{2} \Big).
\end{split}
\end{equation}
where $\mathcal{E} = \sqrt{2\pi \hbar \omega_k / V }$ is the amplitude of the
quantized field.

The first-order approximation in Eq. (5) linearizes the non-linear term.
This is based on the motivation that the resonant frequency of a kerr-cavity
changes as a function of intesity $|\mathbf{E}|^2$. In this case, the resonant
frequency depends on the average photon-number in the kerr-cavity
\begin{equation}
\omega_c^{NL}( N_c = \langle a^{\dagger}a \rangle )  = \omega_c + UN_c.
\end{equation}
Thus the Hamilonian for the single-mode kerr-cavity $\mathcal{H}_a$, the 
bosonic broad-band field $\mathcal{H}_b$, and the free field $\mathcal{H}_f$
is as follows
\begin{align}
\mathcal{H}_a &= \hbar\omega_c^{NL}
    \Big( a^{\dagger}a + \frac{1}{2} \Big),
\\
\mathcal{H}_b &= \hbar\omega_p
    \Big( b^{\dagger}b + \frac{1}{2} \Big),
\\
\mathcal{H}_f &= \hbar \sum_j \omega_j
    \Big( f^{\dagger}_j f_j + \frac{1}{2} \Big),
\end{align}
The discrete cavity mode $\omega_c^{NL}$ and the broad-band mode $\omega_p$
both disipate energy to the free fields $\omega_k$. Thus the interaction
Hamiltonian $\mathcal{H}_{I}$ is as follows
\begin{equation}
\mathcal{H}_I = \hbar g \: (
    a^{\dagger}b + b^{\dagger}a )
    + \hbar \sum_j ( V_j^a f^{\dagger}_j a 
        + V_j^b f^{\dagger}_j b + \mathrm{h.c.} ).
\end{equation}
We work in the purcell regime where $|V_a| \ll g \ll |V_b|$. Note that the
original Fano problem is the limit where $|V_a| \rightarrow 0$.

Deriving the Heisenberg-Langevin equation of motion for the slowly varying
operator $A = a \: e^{i\omega_c^{NL} t}$, and then 
transforming back to the non-slowly varying operator $ a = 
A \: e^{-i\omega_c^{NL} t}$ yields
\begin{align}
\dot{ a } &= -i ( \omega_c^{NL} - i\gamma_c/2 ) a 
    - ig b,
\\
\dot{b} &= -i ( \omega_p - i\gamma_p/2 ) b
    - ig a,
\end{align}
$\gamma_{c,p}$ accounts for both radiative and non-radiative damping as
described in Ref \cite{thakkar2015quantum}. (Note that the above equation
has assumed an evacuated initail reservoir state.)

In the experiment, the broad-band mode is driven by the external
field of a monochromatic laser operating at $\omega$. The spectral 
distribution is derived from the spectral response function, such that,
\begin{equation}
S(\omega) = \frac{1}{\pi} \int_0^{\infty} \mathrm{d}\tau \: 
    e^{i \omega \tau} 
    \langle b^{\dagger}(t_0=0) \: b(t_0 + \tau) \rangle.
\end{equation}
We find that the spectral reponse function can be interpreted as the steady 
state solution of the average photon-number in the broad-band mode rotating 
in the frame of a drive frequency, i.e.
\begin{align}
S(\omega) &= \langle \tilde{b}^{\dagger}_{ss}\tilde{b}_{ss} \rangle
\\
\tilde{b}^{\dagger}_{ss} &= 
    b^{\dagger}_{ss} \: e^{i \omega_L t}; \quad
    \tilde{a}^{\dagger}_{ss} =
    a^{\dagger}_{ss} \: e^{i \omega_L t};
\\
\dot{b}_{ss} &= 0 =
    -i ( \omega_p - i\gamma_p/2 ) b_{ss} - ig a_{ss}
    + iE_{drive}( e^{i \omega_L t} - e^{-i \omega_L t} ).
\end{align}
and $E_{drive}$ is the amplitude of the monochromatic laser operating at
a drive frequency $\omega = \omega_L$. The reduced spectral response yields
\begin{equation}
\begin{split}
\frac{ S(\omega = \omega_L) }{ S(\omega = \omega_L)_{g=0} } &= 
    \frac{\langle \tilde{b}^{\dagger}_{ss}\tilde{b}_{ss} \rangle}
        {\langle \tilde{b}^{\dagger}_{ss}\tilde{b}_{ss} \rangle}_{g=0}
\\
&= \Bigg( 1 +
    \frac{\gamma_c}{\gamma_p}
    \frac{g^2}{(\omega_L - \omega_c^{NL})^2 + (\gamma_c/2)^2}
    \Bigg) \:
    \left\vert \frac{q + \epsilon}{\epsilon + i} \right\vert.
\end{split}
\end{equation}
The first term is the Fano profile, where 
$\epsilon=(\omega_L - \omega_{eff})/(\gamma_{eff}/2)$ and 
$q = (\omega_c^{NL} - i\gamma_c/2 - \omega_{eff})/(\gamma_{eff}/2)$;
$\omega_{eff}$, $\gamma_{eff}$, and $q$ are the Fano parameters. The second
term is the Lorentz distribution due to the line-width broadening of the
discrete state. The spectral distribution reduces to the Fano profile in the
limit $\gamma_c \propto |V_a| \rightarrow 0$.

Since we are interested in single-photon blockade due to the kerr 
non-linearity, $U \propto \chi_{eff}^{(3)}$, we calculate the second order
correlation function $g^{(2)}$ of light transmitted through the kerr-cavity
coupled to the driven broad-band emitter, at the steady state, as a function
of U, i.e.,
\begin{equation}
\begin{split}
g^{(2)}_{ss}(U) &\equiv 
    \frac{ \langle n_a, n_b \vert \:
    \tilde{a}^{\dagger}_0 \: \tilde{a}^{\dagger}_{ss}(U)
    \tilde{a}_{ss}(U) \: \tilde{a}_0 \:
    \vert n_a, n_b \rangle }
    {(\langle \tilde{a}^{\dagger}_0\tilde{a}_0 \rangle)^2}
\\
&= \frac{n_a \langle n_a - 1, n_b \vert \:
    \tilde{a}^{\dagger}_{ss}\tilde{a}_{ss} \:
    \vert n_a - 1, n_b \rangle}
    {n_a^2}
\\
&= \frac{1}{n_a} 
    \frac{g^2}
    {\left\vert \omega_L - \omega_c^{NL}(U) + i\gamma_c/2 \right\vert^2}
    \langle n_a - 1, n_b \vert \:
    \tilde{b}^{\dagger}_{ss}\tilde{b}_{ss} \:
    \vert n_a - 1, n_b \rangle
\\
&= \frac{g^2}{n_a}
    \frac{\langle n_a-1 \vert \: E_{Laser}^2 \: \vert n_a-1 \rangle}
    {\left\vert \big( \omega_L - \omega_c^{NL}(U) + i\gamma_c/2 \big)
    \big( \omega_L - \omega_p + i\gamma_p/2 \big)
    -g^2 \right\vert^2}.
\end{split}
\end{equation}
For single-photon blockade, the photon-number in the cavity is set to one,
such that, $\omega_c^{NL}( U, N_c = 1 )  = \omega_c + U$. Importantly, the
amplitude of the Laser field to populate the bare ($N_c=0$) cavity mode 
with a single photon at steady state is such that
\begin{equation}
\begin{split}
\langle n_a=1 \vert \:
    \tilde{a}^{\dagger}_{ss} \tilde{a}_{ss} \:
    \vert n_a=1 \rangle 
    &= n_a = 1
\\
&= \frac{ g^2 \: 
    \langle n_a \vert \: E_{Laser}^2 \: \vert n_a \rangle}
    {\left\vert \big( \omega_L - \omega_c^{NL}(N_c=0) + i\gamma_c/2 \big)
    \big( \omega_L - \omega_p + i\gamma_p/2 \big)
    -g^2 \right\vert^2}.
\end{split}
\end{equation}
thus, it follows that
\begin{equation}
\langle n_a \vert \: E_{Laser}^2 \: \vert n_a \rangle = n_a
    \left\vert 
    ( i\gamma_c/2 )( \omega_c - \omega_p + i\gamma_p/2 ) -g^2 
    \right\vert^2
    / g^2.
\end{equation}

\section{Previous work}
Previous work involved the study of individual split ring resonator (SRR) 
cavities arranged in a dimer or trimer array. ref[]  We invesitgated the 
coupling strength between the coupled SRR dimer as a function of the
two-dimensional in-plane rotaion (Fig. 1), and the three-dimensional 
out-of-plane tilt (Fig 2). This study was the first to characterize the latter 
degree of freedom, where we found that...

This work is currently under review at ACS Optical Applied Materials.

\newpage
\bibliography{ref_database}
\bibliographystyle{unsrt}

\end{document}
